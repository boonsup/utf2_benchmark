% =====================================================================
% Supplementary Section S1: Kernel Simulation Architecture
% =====================================================================

\clearpage
\section*{Supplementary Section S1: Kernel Simulation Architecture}

\noindent
Section~S1 presents the reference simulation kernel used to reproduce the
UTFv2 computational experiments. This kernel serves as the deterministic backbone
for the full pipeline, integrating the $\hat{T}$, $\hat{D}$, and $\hat{F}$
operators within a modular simulation loop.

The implementation employs explicit time‐stepping with adaptive control
to ensure numerical stability during chaotic amplification sweeps.
All results are initialized with fixed random seeds (\texttt{seed = 42})
and checkpointed for full replay reproducibility.

\begin{figure}[h!]
    \centering
    \includegraphics[width=\textwidth]{figures/supplementary_S1_kernel_architecture.png}
    \caption{
        \textbf{S1. Kernel Execution Flow.}
        The UTFv2 simulation kernel implements three interacting operators:
        $\hat{T}$ (energy transmutation), $\hat{D}$ (decoherence diffusion),
        and $\hat{F}$ (chaotic transfusion). The figure depicts the sequence
        of operations and inter‐operator data exchange, highlighting the
        adaptive $\Delta t$ controller for maintaining stable chaotic evolution.
    }
    \label{fig:s1_kernel}
\end{figure}

\vspace{1em}
The corresponding CSV file
(\texttt{data/supplementary\_S1\_kernel.csv})
contains the logged evolution of the $\alpha$, $\beta$, and $\lambda$
coefficients used for the kernel performance benchmarks.
