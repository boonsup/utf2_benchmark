% =====================================================================
% Supplementary Section S3: Quantum–Classical Coupling Test
% =====================================================================

\clearpage
\section*{Supplementary Section S3: Quantum–Classical Coupling Test}

\noindent
Section~S3 explores the coupling consistency between the quantum operator
($\hat{D}$, decoherence domain) and the classical chaotic operator
($\hat{F}$, transfusion domain). The hybrid bridge enforces stability across
the decoherence threshold via the coupling parameter $\eta$, ensuring
that energy leakage remains bounded.

The test configuration follows:
\[
\eta_c = \frac{\partial E_{\mathrm{q}}}{\partial t}
         \bigg/ \frac{\partial E_{\mathrm{c}}}{\partial t},
\quad
\text{where } E_{\mathrm{q}}, E_{\mathrm{c}} \text{ denote quantum and classical energies.}
\]

\begin{figure}[h!]
    \centering
    \includegraphics[width=\textwidth]{figures/supplementary_S3_coupling_variation.png}
    \caption{
        \textbf{S3. Quantum–Classical Coupling Variation.}
        The normalized energy exchange between the decoherence ($\hat{D}$)
        and chaos ($\hat{F}$) domains, evaluated for different coupling
        strengths $\eta$. The equilibrium plateau corresponds to the
        optimal $\eta_c$ at which energy conservation and stability are
        simultaneously satisfied.
    }
    \label{fig:s3_coupling}
\end{figure}

\vspace{1em}
The numerical data supporting Fig.~\ref{fig:s3_coupling} are stored in
\texttt{data/supplementary\_S3\_coupling.csv}, including the evaluated
$\eta$, $\lambda$, and energy drift metrics.
